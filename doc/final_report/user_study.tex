We conducted a post-implementation user study of our application, with the intention of answering 2 main questions.
\begin{enumerate}
	\item Does the application communicate well with the user? That is, can a user easily understand the information laid out by the app?
	\item Does the privacy enhancing measures like data summarization really make a difference in users' decision to share data with third parties?
\end{enumerate}

\subsection{User study setup}

\textit{Survey questions}: The survey involved presenting the users with screenshots of the application, and collecting their responses to determine whether the app meets its goals. To measure how well users understand the information presented in the app, we asked the following questions:
\begin{enumerate}
	\item Who stores the data generated by the home thermostat?
	\item Who is requesting access to hourly averages of the thermostat data?
\end{enumerate}

These additional questions were asked to ascertain conditions under which users are more willing to share their data with third parties.
\begin{enumerate}
	\item In the given example, the data requester is asking only for hourly averages of your thermostat data, and not the entire data collected by the device. Does this make you more willing to share the data?
	\item If a third party data requester has endorsements from entities known to you, will you be more willing to share your data with them?
\end{enumerate}

\textit{Test platform}: To gain feedback about the app from a wide audience, we used Amazon mechanical turk for conducting the study~\cite{mturk}. We created a survey in the platform using app screenshots and the questions listed above. We offered \$0.15 for each participant and collected responses of 100 participants over the course of a few hours.

\subsection{User study results}
A summary of the responses by the users is shown in Fig.~\ref{fig:user_study}. The data shows that almost 70\% of the users were aware of the source of their data. More importantly, 90\% of the users correctly identified the data requester. Around 50\% of the users were more willing to share their data in case of endorsements or data summarizations. This shows that users do care about privacy and third-party trust mechanisms. Incorporating such features will be essential to collect data from maximum number of users in an IoT system.

We understand that this preliminary user study is not comprehensive enough to cover all privacy-protecting aspects discussed in Davies et al. However, the initial results suggest that privacy is important for users, and further work on user facing components of IoT systems are absolutely necessary. We plan to conduct more sophisticated user studies along with the planned future enhancements to our application.