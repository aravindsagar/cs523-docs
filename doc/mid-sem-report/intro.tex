Smaller and more power efficient processors enables almost everything in our life to collect information and access internet. This gave rise to the idea of “Internet of Things” (IoT).

IoT has seen a rapid growth in recent years, as benefits of a smarter world around us can be huge. On the other hand, rapid expansion of IoT devices also brings new technical and privacy related challenges related to them. Apart from the technical challenges like storage and processing of vast amounts of data, and building scalable communication channels, we also face a new era of privacy-related challenges arising from devices collecting a multitude of personal data. Assuring users of the privacy of their data is an essential step towards making IoT a commercial success \cite{davies}.

The primary issues with currently available commercial IoT ecosystems is that third-party access control mechanisms of user data is centralized, and that permissioning is too coarse-grained. The first issue requires users to trust a single party with all their data, and at the same time introduces a single point of failure. The second issue provides more personal data than is required, to third parties. Several frameworks have been proposed to address these challenges \cite{campbell} \cite{davies}. These systems aim to address the privacy concerns by putting the user directly in control of their data. Decentralizing access control mechanisms, fine-grained permissioning, and explicit user consent from the user to share just the right amount of data with third parties are common themes found in such frameworks.

While these frameworks can drastically improve the privacy of users in IoT systems, there are still gaps that need to be filled, in order to truly use these systems in real-world. One such gap is the daunting technicality of these systems. We cannot assume that a normal user can deal with cryptographic primitives, setting up servers, data aggregation and summarization, and a variety of other knowledge required to interact with such systems. This is the gap that we are trying to address in our work.

We propose building a user facing application and a corresponding server component that can help even non tech savvy users exploit privacy-aware, distributed, and cryptographically secure IoT systems. For the purposes of our implementation, we will assume that such an IoT system exists, and emulate the actions of that system. The app will help the users to visualize all the data owned by them, and manage fine-grained data access to third parties. Each user will have a dedicated instance of the server component, which will act as a privacy-mediator\cite{davies}, that is, enforce the data access control policies.

Our main contributions include:
\begin{enumerate}
	\item We aim to build a user friendly application that puts the user in control of their data, and lets them understand and allow just the right amount of data to be available for third parties.

	\item For the server component, we aim to make it secure from external attacks, and also design it in such a way that users have the freedom to run their own instance of the server or get an instance of it in commercial servers.
\end{enumerate}
