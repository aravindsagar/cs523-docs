Smaller and more power efficient processors provide more options in our life to collect information and access internet. This gave rise to the idea of ``Internet of Things'' (IoT).

We have observed a rapid growth of IoT market~\cite{idc} in recent years, as people started to adopt this new technology. On the other hand, rapid expansion of IoT devices also brings new technical and privacy related challenges. Apart from the technical challenges like storing and processing large volume of data, and building scalable communication channels, we also face a new era of privacy-related challenges arising from devices collecting a multitude of sensitive data. Assuring privacy of these data to users is an essential step towards making IoT a commercial success~\cite{davies}.

There exists several issues with currently available commercial IoT ecosystems. For instance, the access control policies are enforced by a central entity and the policies themselves are often too coarse-grained. The first issue forces users to trust a single party with all their data, and at the same time this centralized design inherently leads to a single point of failure. The second issue exposes more potentially sensitive data  to data requesters. Several frameworks have been proposed to address these challenges~\cite{campbell,davies}. These systems aim to address the privacy concerns by putting the user directly in control of their data. Decentralizing access control mechanisms, fine-grained permissioning, and explicit user consent from the user to share just the right amount of data with third parties are common themes found in such frameworks.

While these privacy-preseving frameworks can drastically improve the privacy protection of user data in IoT systems, these theoretical models overlook some key aspects that must be addressed before the implementation. One such aspect is a gap between the expectation and reality of the end user's technical knowledge. We cannot assume that a normal user can comprehend cryptographic primitives, can set up back-end components correctly, perform data aggregation and summarization, and etc. Our work directly addresses this gap.

We propose a user-friendly IoT framework, composed of front-end mobile application and a corresponding server component. This framework can help even non-tech savvy users to  utlize privacy-aware, distributed, and cryptographically secure IoT systems. We first investigate a transparent, decentralized blockchain-based IoT data-sharing model.~\cite{campbell} Our app will help the users to visualize all the data owned by them, and manage fine-grained data access controls to data requesters. Each user will have a dedicated instance of the server component, which will act as a privacy-mediator~\cite{davies}, that enforces the data access control policies.

Our main contributions include:
\begin{enumerate}
	\item We aim to build a user friendly application that puts the user in control of their data, and lets them understand and allow just the right amount of data to be available for third parties.

	\item For the server component, we aim to make it secure from external attacks, and also design it in such a way that users have the freedom to run their own instance of the server or get an instance of it in commercial servers.
\end{enumerate}