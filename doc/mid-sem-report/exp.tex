We started out by experimenting with the UI of the application. We are currently developing an Android app, and apps for other mobile platforms like iOS is left as future work.

We made an initial version of the UI mockups, and ran a user survey to iterate over the design. The current design is shown in ``Fig.~\ref{fig:screens}''. These images show only some of the important screens in the app, and are not exhaustive. The capabilities of the app include having an overall view of the devices and data (Fig.~\ref{fig:screen1}), seeing and editing details pertaining to a particular device (Fig.~\ref{fig:screen2}), receiving and acting upon data requests (Fig.~\ref{fig:screen4}), and viewing and managing trusted parties.

User research shows that the following design choices enhance the experience.

\begin{itemize}
	\item Overall view of the devices and data is very helpful.
	\item Notifications screen is well done because users can see whether the requester is endorsed and act on the notification from the same screen.
	\item Users also feel more comfortable with IoT devices when a privacy-control mechanism like this app is present.
\end{itemize}

We also identified some major areas of concern.

\begin{itemize}
	\item Users need more control over what data is stored. In particular, they might want to delete data between certain time periods.
	\item Users want to minimize the data shared with third parties, and hence each device should expose various spatial and temporal summarizations. 
	\item Guarantee of anonymity is a must when sharing sensitive data with third parties, especially for analytics purposes.
\end{itemize}