As evident from the previous sections, effectively addressing privacy concerns about the data generated by IoT devices is essential for the success of IoT. There are some proposed architectures which tackles this problem using user-oriented and decentralized systems. However, there's a lack of a user-facing component in such systems, which can effectively make the complex architectures usable to everyone. Our work is aimed at tackling this problem, which is "to build a user-facing component for a privacy-aware distributed user-centric IoT system (such as the blockchain based system described in \cite{campbell}), which improves the user friendliness of the system while maintaining strong security and privacy guarantees."

To accomplish this, we build the user-facing component for the blockchain based IoT architecture described in "World of empowerd IoT users \cite{campbell}". Since that system is still in development, we concentrate on the user facing components of the system, and simulate some of the external components.

The proposed work consists of 2 components:

\begin{enumerate}
	\item A smartphone app to manage IoT devices and data: Smartphones are becoming the preferred devices for internet access \cite{statcount}. Hence we build a smartphone app with which users can manage their IoT devices and data. The functionality provided by the app includes:

	\begin{enumerate}
		\item A centralized view of IoT generated data owned by them.

		\item A detailed view including the data source, access permissions and other details of this data.

		\item Permission request notifications from data requesters, and a way to accept or deny them.

		\item View, add, remove trusted parties.
	\end{enumerate}

	This is not a comprehensive list of the functionality, as we expect to add more as the project progresses. It also hides away some complexity from the users, like managing secure connection with the server described below, and managing user's private key etc. We also focus on the user-friendliness of the app, and expect to conduct user-research to find out what capabilities do users expect to find in such an app.
	
	\item A trusted server which manages the access control: This is the component which actually manages access control and capability issuance in accordance to the user's wishes. This is similar to cloudlets described in \cite{davies}, but with some key differences. First, we intend to leverage hardware technologies like Intel SGX to ensure that the server application can run on untrusted hosts. Second, this app does not store data within it; it simply interacts with the data owner, data sources and data requesters in accordance to the protocol outlined in \cite{campbell}.

	

\end{enumerate}