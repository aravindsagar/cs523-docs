As evident from the previous sections, effectively addressing privacy concerns about the data generated by IoT devices is essential for the success of IoT. There are some proposed architectures which tackles this problem using user-oriented and decentralized systems. However, there's a lack of a user-facing component in such systems, which can effectively make the complex architectures usable to everyone. Our work is aimed at tackling this problem, which is ``to build a user-facing component for a privacy-aware, distributed, secure and user-centric IoT system (such as the blockchain based system described in~\cite{campbell}), which improves the user friendliness of the system while maintaining strong security and privacy guarantees."

To accomplish this, we build the user-facing component for the blockchain based IoT architecture described in ``World of Empowered IoT Users~\cite{campbell}''. Since that system is still in development, we concentrate on the user facing components of the system, and simulate some of the external components.

The proposed work consists of two components:

\begin{enumerate}
	\item A smartphone app to manage IoT devices and data: Smartphones are becoming the preferred devices for internet access~\cite{statcount}. Hence we build a smartphone app with which users can manage their IoT devices and data. The functionality provided by the app includes:

	\begin{enumerate}
		\item A centralized view of IoT generated data owned by them.

		\item A detailed view including the data source, access permissions and other details of this data.

		\item Permission request notifications from data requesters, and a way to accept or deny them.

		\item View, add, remove trusted parties.

		\item Provide user's (data owner's) public key to data requesters that provide personalized services to the user. This is not applicable for users providing anonymous data for purposes like analytics, where a different public key of the user is exposed to the data requesters.

	\end{enumerate}

	This is not a comprehensive list of the functionality, as we expect to add more as the project progresses. It also hides away some complexity from the users, like managing secure connection with the server described below, and managing user's private key etc. We also focus on the user-friendliness of the app, and expect to conduct user-research to find out what capabilities do users expect to find in such an app.
	
	\item A trusted server which manages the access control: This is the component which actually manages access control and capability issuance in accordance to the user's wishes. This is similar to cloudlets described in~\cite{davies}, the key difference being that this server does not store data within it; it simply interacts with the data owner, data sources and data requesters in accordance to the protocol outlined in~\cite{campbell}. Interaction between server and data owner happens via the app described above. Interaction with other parties happen through a messaging layer, similar to the one outlined in~\cite{campbell}. The server ensures security in its communications using cryptographic primitives as outlined in the communication protocol.

	The server will also convert the technical specification of the data model and aggregation methods of each IoT device into user friendly texts, and uses these user friendly texts while pushing data access requests to the data owner's mobile app. We assume that the data model is agreed upon by the device and data sources. Creating required drivers for IoT devices is out of scope of this work.

	As we are focusing on user friendliness of the solution, and setting up a server may not be forte of everyone, a commercial implementation of our solution should ideally provide the users an option to automatically allocate a VM for the user in a commercial server. The users will of course have an option to use their own server instead. Another option is to have a `hub' for each user, which will run the server process, but this solution may not be scalable for the user. We do not focus on this aspect, and instead focus on an implementation of the server which can be used for any of these purposes.
\end{enumerate}