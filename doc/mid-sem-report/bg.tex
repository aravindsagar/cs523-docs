\subsection{Internet of Things}
TODO
(Can potentially include the paper exposing security holes in Samsung SmartThings)

\subsection{Security and provacy in IoT (Blockchain paper)}
TODO

\subsection{Intel SGX}
TODO

\subsection{Related Works}
Davies et. al. points out that privacy concerns could be a major factor preventing wide-spread adoption of IoT devices \cite{davies}. Over-centralization of IoT systems is identified as a critical obstacle to eliminating concern over privacy. They propose a privacy mediator which sits in between IoT devices/data and outside world. This model includes a trusted 'cloudlet', which could be a local hub or a trusted server at the edge of the cloud. They also emphasize some important characteristics that a privacy-aware IoT system should have, like exposing summarized data, user anonymity, control over inferred sensors, and ease of use. However, the model's limitation is that data storage has to happen in the sensor or the cloudlet - this could be limiting when the number of sensors increase.

Data mining and clustering of IoT related articles expose major problem areas in IoT \cite{zhang}. App over-privilege, environment mistrust, LAN mistrust and weak authentication are identified as some of the problem areas that needs work. They also conclude that permissioning needs to be more fine-grained, and recommend better standards and widely-applicable systems.