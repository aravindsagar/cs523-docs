\subsection{Internet of Things}
The term Internet of Things (IoT) originated more than 15 years ago, when it was used to describe the work of the Auto-ID Labs at the
Massachusetts Institute of Technology (MIT) on networked radio-frequency identification (RFID) infrastructures~\cite{atzori}. The definition has since expanded to beyond the scopr of RFID technologies. A lot of modern definitions have been proposed, one of them being ``a global infrastructure for the Information Society, enabling advanced services by interconnecting (physical and virtual) things based on existing and evolving interoperable information and communication technologies''~\cite{itu}.

The applications of IoT are diverse, ranging from smart industries to smart homes. In smart home area, thermostats, security systems, and energy management systems are particularly growing fast. In this paper, we focus on IoT technologies related to individual users, and these mostly fall under the smart home category, though the general principles are applicable in all areas of IoT.

Recent years have seen a rapid growth of smart devices available commercially. Ecosystems like Samsung SmartThings, Google Home and Apple Homekit are competing to become the primary player in personal IoT devices space, and for good reason -  it is estimated that IoT could grow into a market worth \$7.1 trillion by 2020~\cite{idc}.

However, the rapid expansion of IoT market also means that many issues have been overlooked, and still need resolution. Some of the relevant issues are presented in `Related Works'.

\subsection{Security and privacy in IoT (Blockchain paper)}
IoT data security is crucial in protecting user privacy. In the paper ``World of Empowered IoT Devices" \cite{campbell}, a IoT data management system is proposed. This provides a user-centric, scalable, distributed and privacy-aware system to share data between the users (data owners) and third party data requesters.

In the data management protocol of this framework, user data is stored in trusted third-party entities, known as “data sources”. IoT users are known as “data owners” to the data their devices collected. Entities who are trying to get access to user data are known as “data requesters”. 

The data owner is the only party who can grant data requesters the right, called “capability”, to access its proprietary data. When the data requester needs data access, it contacts the data owner for the capability. After acquiring the capability, the data requester will access data at data source. Throughout the process, the data owner has the control its data.

To ensure data transparency, all data transactions are recorded using BlockChains. After a successful data transaction, the data server broadcasts the transaction and the publishers update their BlockChain Record. 

\subsection{Related Works}
Davies et. al. points out that privacy concerns could be a major factor preventing wide-spread adoption of IoT devices~\cite{davies}. Over-centralization of IoT systems is identified as a critical obstacle to eliminating concern over privacy. They propose a privacy mediator which sits in between IoT devices/data and outside world, and validates permissions according to the access control policies before any data is sent out. This model includes a trusted `cloudlet', which could be a local hub or a trusted server at the edge of the cloud. They also emphasize some important characteristics that a privacy-aware IoT system should have, like exposing summarized data, user anonymity, control over inferred sensors, and ease of use. However, the model's limitation is that data storage has to happen in the sensor or the cloudlet - this could be limiting when the number of sensors increase. The model also does not discuss any framework for how to identify legitimate third party applications, and how to make sure that the request originates from the same application as it is claiming to be from.

Data mining and clustering of IoT related articles expose major problem areas in IoT~\cite{zhang}. App over-privilege, environment mistrust, LAN mistrust and weak authentication are identified as some of the problem areas that needs work. They also conclude that permissioning needs to be more fine-grained, and recommend better standards and widely-applicable systems.
