\subsection{Internet of Things}
TODO
(Can potentially include the paper exposing security holes in Samsung SmartThings)

\subsection{Security and privacy in IoT (Blockchain paper)}
Nowadays, smart devices like Android Wearables and smart phones are collecting information in every aspect of device owners and generate vast amount of data. While the sheer size and growing rate of the data present a challenge to store and process them, keeping the data secure is equally challenging and important.

The paper "World of Empowered IoT Devices" proposes a system that aims to address these problems. In this system, user data is stored in volunteer data servers. These data servers let the data owners to name trusted entities who can access the proprietary data. A data requester can only get access to the data if it has the permission given by the data owner(IoT device user). 

To ensure data transparency, all data transactions are recorded using BlockChains. A Publisher-Subscriber configuration is used to keep track and distribute the BlockChain nodes. After a successful data transaction, the data server broadcasts the transaction and the subscribers update their BlockChain Record.

Using the combination of user-granted data access and BlockChain, the system proposed by "World of Empowered IoT Devices" provides a secure and scalable way to manage the ever-growing IoT user data.

\subsection{Intel SGX}
TODO

\subsection{Related Works}
Davies et. al. points out that privacy concerns could be a major factor preventing wide-spread adoption of IoT devices \cite{davies}. Over-centralization of IoT systems is identified as a critical obstacle to eliminating concern over privacy. They propose a privacy mediator which sits in between IoT devices/data and outside world. This model includes a trusted `cloudlet', which could be a local hub or a trusted server at the edge of the cloud. They also emphasize some important characteristics that a privacy-aware IoT system should have, like exposing summarized data, user anonymity, control over inferred sensors, and ease of use. However, the model's limitation is that data storage has to happen in the sensor or the cloudlet - this could be limiting when the number of sensors increase.

Data mining and clustering of IoT related articles expose major problem areas in IoT \cite{zhang}. App over-privilege, environment mistrust, LAN mistrust and weak authentication are identified as some of the problem areas that needs work. They also conclude that permissioning needs to be more fine-grained, and recommend better standards and widely-applicable systems.